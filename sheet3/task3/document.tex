\documentclass{article}
\usepackage[utf8]{inputenc}
\usepackage{graphicx}
\usepackage{amsmath}
\usepackage{booktabs}
\usepackage{textcomp}
\usepackage{multirow}
\usepackage{color}
\usepackage{listings}

\lstset{
	numbers=left,
	basicstyle=\small,
	tabsize=3,
	showspaces=false,
	showtabs=false,
	showstringspaces=false
}

\begin{document}
\title{Assignment 3.3}
\author{Christian Müller, Ralph Krimmel \& Sebastian Albert}
\maketitle
\section*{(a)}
A NAS gateway can make use of an existing SAN and share it with other applications.
However, a backup-to-disk SAN is likely to be optimized for the read-write-behaviour of backups (i.~e., many large sequential writes and almost no reads) and thus not be very well-suited for file sharing purposes that a NAS is usually used for.
\section*{(b)}
Possible drawbacks of unsynchronized TCP window sizes:
\begin{itemize}
\item Too small size on sender side: Capabilities not used efficiently.
\item Too large size on sender side: Buffer overflows, leading to dropped packets, resending necessary
\item Overfull buffers may even decrease the performance for other users that are correctly configured.
\end{itemize}
\section*{(c)}
Jumbo frame support requirements:
\begin{itemize}
\item all intermediate hops should support Jumbo Frames
\item MTU size should be equal
\item Slowdown may be caused by different MTU sized (especially by no Jumbo frame support at all by some device in between) - packets get split up and buffers congested; in case of a bad combination of different MTU sizes, speed can even decrease by up to 50\% (consider an \emph{almost} fitting MTU size)
\end{itemize}
\section*{(d)}
NAS performance can be increased by finding the bottleneck within the many possibilities affecting the performance:
\begin{itemize}
\item Number of hops (try to reduce if bottleneck)
\item Authentication with a directory service
\item Retransmission (check window sizes)
\item Overutilized routers and switches (upgrade hardware or add redundant paths)
\item File/directory lookup and metadata requests (add NAS heads)
\item Overutilized NAS devices (add more)
\item Overutilized clients (their problem)
\end{itemize}
In general, availability can be achieved by redundancy.
\begin{itemize}
\item outside the NAS: keep redundant paths (switches, cables, ...)
\item behind the NAS: redundant storage like RAID
\item the NAS itself: keep a failover NAS head
\end{itemize}
\end{document}
