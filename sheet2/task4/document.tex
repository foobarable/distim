\documentclass{article}
\usepackage[utf8]{inputenc}
\usepackage{graphicx}
\usepackage{amsmath}
\usepackage{booktabs}
\usepackage{textcomp}
\usepackage{multirow}
\usepackage{color}
\usepackage{listings}

\lstset{
	numbers=left,
	basicstyle=\small,
	tabsize=3,
	showspaces=false,
	showtabs=false,
	showstringspaces=false
}

\begin{document}
\title{Aufgabenblatt 1}
\author{Christian Müller, Ralph Krimmel \& Sebastian Albert }
\maketitle
\section*{(a)}
Which application benefits the most from bypassing the write cache? Any application for which the only or main purpose is writing large amounts of data. It does not benefit from a shorter return time from write operations and the cache is too small to hold all the written data. Example: Copying files.
\section*{(b)}
Cache coherence mechanisms maintain the notion of a state for each particular cache line. The behaviour of the cache is determined by the current state.
\begin{itemize}
\item \emph{Modified}: has been modified in the cache and needs to be written to the underlying memory (also called \emph{dirty})
\item \emph{Shared}: present in cache, unmodified, safe
\item \emph{Invalid}: 
\end{document}
