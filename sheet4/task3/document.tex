\documentclass{article}
\usepackage[utf8]{inputenc}
\usepackage{graphicx}
\usepackage{amsmath}
\usepackage{booktabs}
\usepackage{textcomp}
\usepackage{multirow}
\usepackage{color}

%use (a) for numbering subsections
\renewcommand*\thesubsection{(\alph{subsection})}
\title{Aufgabenblatt 4}
\author{Christian Müller, Ralph Krimmel \& Sebastian Albert }

\begin{document}

\maketitle

\section*{Assignment 3 - Backup and Archive}

\subsection{Propose a backup and recovery solution for the described company!}

\subsection{Discuss the security conerns in backup environments?}

\subsection{List and explain the considerations in tape as backup technology!
				What are the challenges in this environment?}

\subsection{Describe the benefits of using ``virtual tape libaries'' over ``physical tapes''!}
	Virtual tape libaries (VTL) have some key advantages over pyhsical tapes.
	These boil down to the fact,
	that disks are used in stead of tapes
	and thereby eliminating the problems related to using tapes.
	Tapes suffer from tear and wear and the so called shoe-shine-effect,
	which reduces their reliabilty.
	These both features allow faster backup and recovery
	then pyhsical tapes.
	Additionally VTLs offer support for data replication over IP networks,
	so that inexpensive offsite replication is available.
	As VTLs come preconfigured,
	they are easier to integrate than backup-to-disk services,
	especially if a VTL substitutes physical tapes.

\subsection{Describe the data deduplication methods and distinguish source-based and target-base data deduplication!}

\end{document}
