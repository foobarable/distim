\documentclass{article}
\usepackage[utf8]{inputenc}
\usepackage{graphicx}
\usepackage{amsmath}
\usepackage{booktabs}
\usepackage{textcomp}
\usepackage{multirow}
\usepackage{color}


\begin{document}
\title{Aufgabenblatt 1}
\author{Christian Müller, Ralph Krimmel \& Sebastian Albert }

\maketitle


\section{Assignment 1 - Data Center Infrastructure}

\subsection{a}
	The core elementns of a data center are:\\
	%TODO Add picture from module01 p. 14
	\begin{description}
		\item[Application] \hfill \\
			Software that provides the logic for computing operations.
			It is characterized by its I/O profile,
			which describes how the application reads and writes data.
			Typical categories are read or write intensive applications
			or ones that have more sequential or random read and write operations.
			Another factor is the typical I/O size.i
			Applications can be isolated by virtualization,
			which avoids conflicts and can enhance data protection.\\
			Examples are: ERP, authentication or backup.

		\item[Database management system (DBMS)] \hfill \\
			Databases store structured data in tables that have can have relations with each other.
			DBMS controls the creation, maintance and use of the databases.
			These systems can be querried by applications for certain information.
			
		\item[Host or Compute] \hfill \\
			A ressource that runs an application with the its computing components.
			It consist of a physical machine and the operating system.
			They enable the application to run by supplying CPU, memory and I/O devices on the hardware side
			and operating system, device driver, file system and volume manager on the software side.\\
			Hosts can be virtualized to enhance the usability of the physical entity.
			Examples for hosts are servers, clusters, laptops and desktops.

		\item[Network] \hfill \\
		\item[Storage] \hfill \\
	\end{description}




\subsection{a}

\end{document}
